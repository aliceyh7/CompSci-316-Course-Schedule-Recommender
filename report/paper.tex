\documentclass{article}

% packages
\usepackage[letterpaper, top=1in, bottom=1in, left=1in, right=1in]{geometry}
\usepackage[utf8]{inputenc}
\usepackage[english]{babel}
\usepackage{amsmath} 
\usepackage{amssymb}
\usepackage{mathtools}        % for extra stuff like \coloneqq
\usepackage{mathrsfs}         % for extra stuff like \mathsrc{}
\usepackage{centernot}        % for the centernot arrow 
\usepackage{bm}               % for better boldsymbol/mathbf 
\usepackage{enumitem}         % better control over enumerate, itemize
\usepackage{hyperref}         % for hypertext linking
\usepackage{fancyvrb}         % for better verbatim environments
\usepackage{newverbs}         % for texttt{}
\usepackage{xcolor}           % for colored text 
\usepackage{listings}         % to include code
\usepackage{lstautogobble}    % helper package for code
\usepackage{parcolumns}       % for side by side columns for two column code
\usepackage{fancyhdr}         % for headers and footers 
\usepackage{lastpage}         % to include last page number in footer 
\usepackage{parskip}          % for no indentation and space between paragraphs    
\usepackage[T1]{fontenc}      % to include \textbackslash
\usepackage{footnote}
\usepackage{etoolbox}

% for custom environments
\usepackage{tcolorbox}        % for better colored boxes in custom environments
\tcbuselibrary{breakable}     % to allow tcolorboxes to break across pages

% figures
\usepackage{pgfplots}
\pgfplotsset{compat=1.18}
\usepackage{float}            % for [H] figure placement
\usepackage{tikz}
\usepackage{tikz-cd}
\usepackage{circuitikz}
\usetikzlibrary{arrows}
\usetikzlibrary{positioning}
\usetikzlibrary{calc}
\usepackage{graphicx}
\usepackage{caption} 
\usepackage{subcaption}
\captionsetup{font=small}

% for tabular stuff 
\usepackage{dcolumn}

\usepackage[nottoc]{tocbibind}
\pdfsuppresswarningpagegroup=1
\hfuzz=5.002pt                % ignore overfull hbox badness warnings below this limit

% Page style
\pagestyle{fancy}
\fancyhead[L]{CS 316, Databases}
\fancyhead[C]{}
\fancyhead[R]{Fall 2024} 
\fancyfoot[C]{\thepage / \pageref{LastPage}}
\renewcommand{\footrulewidth}{0.4pt}          % the footer line should be 0.4pt wide
\renewcommand{\thispagestyle}[1]{}  % needed to include headers in title page

\begin{document}

\title{Coursers}
\author{Daniel Zeng, Muchang Bahng, Abhishek Chataut, Alice Hu, Jasper Hu}
\date{Fall 2024}

\maketitle

\section{Introduction} 

\section{Changes}
We discovered that we can use the Duke Developer Console API to get the course data that we require instead of web scraping from the Duke course catalog. The API provides a structured way to access all course data, which is more efficient and easy to parse with JSON. However, even the API does not cleanly return the prerequisites for a class, so we may need to fall back on web scraping in the future if the API does not meet our needs.

We updated our tech stack to include Flask as our back-end, utilizing the starter code from the example Amazon project. Additionally, we plan to use SQLAlchemy to add PostgreSQL support for Flask, as demonstrated in the example project from the following GitLab repository: \href{https://gitlab.oit.duke.edu/compsci316/mini-amazon-skeleton-24-fall/-/tree/main?ref_type=heads}{Mini Amazon Skeleton}.


\section{Methodology}

\end{document}
